\section{Einleitung}
    - Theorie aufstellen [max. 2 Seiten] 

\section{Theoretischer Hintergrund}

\subsection{stadt.werk}
    - Informationen zum Unternehmen [max. 2 Seiten]

\subsection{Wissensmanagement}
    - Bedeutung von Wissensmanagement [2 Seiten]
    - Wissensmanagement im Wandel der Zeit [1 Seite]
    - Wissensmanagement im Kontext von Corona und Homeoffice [1-2 Seiten]

\subsection{Elasticsearch}
    - kurze Erläuterung zu Elasticsearch [1 Seite]

\subsubsection{Indices, Types, Documents, Fields}
    - Wie sind Elemente der Elasticsearch aufgebaut? [2-3 Seiten]
    
\subsubsection{Scoring}
    - Boolean model, TF/IDF, vector space model [4-5 Seiten]
    
\subsection{Vergleich mit anderen Datenbanksystemen}
    - Relationale Datenbanken vs Dokumentenorientierte Datenbanken [3-4 Seiten]
    - Warum Elasticsearch für dieses oder andere Projekte? [2-3 Seiten]
    
\section{Umsetzung}

\subsection{Idee}
    - Beschreibung des Projektes [1-2 Seiten]

\subsection{Wahl der Technologien}
    - Node.js, npm, JavaScript, Docker, Kibana, Postman etc. [4-5 Seiten]

\subsection{Schwierigkeiten}
    - Schwierigkeiten, auf die ich während der Umsetzung gestoßen bin [1-2 Seiten]
    
\subsection{Ergebnis der Umsetzung}
    - Vorstellung Code und gewählte Infrastruktur [4 Seiten]

\subsection{Überprüfung der aufgestellten Theorie}
    - Vorgehensweise Überprüfung mittels Fragebogen [2-3 Seiten]

\section{Ergebnisse}
    - Ergebnisse des Fragebogens aufzeigen [4-6 Seiten]

\section{Fazit}
    - Ergebnisse des Fragebogens analysieren und bewerten [2-3 Seiten]
    - aufgestellte Theorie verifizieren oder falsifizieren [2-3 Seiten]

\section{Methodik}
    - gewählte Methodik begründen und kritisieren [2-3 Seiten]
    